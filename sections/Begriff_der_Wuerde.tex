\documentclass[../main.tex]{subfiles}
\begin{document}

\subsection{Würde: Methodik}
Würde Würde wird in zwei Unterkategorien eingeteilt:
\begin{enumerate}
	\item \textbf{Kontingente Würde} bezeichnet die Würde, die einem  \textit{nicht} als Mensch zukommt, sondern aufgrund einer Leistung. Beispiel dafür ist die Würde der Richterin.
	\item \textbf{Inhärente Würde} bezeichnet die Würde, die uns als Mensch zukommt, weil wir ein Mensch sind. Sie kann verletzt, jedoch nicht verloren werden. 
\end{enumerate}

Bei der Analyse von Würde ist es essentiell, dass ein Würdebegriff gewisse Aspekte der Würde abdecken muss;
\begin{enumerate}
	\item Paradigmatische Fälle der Würdeverletzung: Es muss klar sein, was paradigmatische Fälle der Würdeverletzung zu solchen macht. Das heisst, es muss klar sein, warum Folter die Würde des Opfers verletzt. 
	\item Der Zusammenhang mit Begriffen wie Grundrechte und Menschenrechte muss sichergestellt sein. 
\end{enumerate}

\subsection{Würde als Lebensideal}
Erstmals aufgegriffen wurde die Würde von Poseidonios von Rhodos (135-51 v.Chr.) und Cicero (106-43 v.Chr.). Gemäss ihnen ist sie ein innerer Wert, der allen Menschen zukommt. Somit hat die Würde bereits einen inherenten Charakter, der sich aber noch nicht ganz deutlich zeigt, da gemäss diesen alten griechischen Philosophen diese Würde zwar bereits entkoppelt ist von unserem gesellschaftlichen Ansehen, jedoch immer noch an unsere Vernunft (in ihrem Verständnis unsere Fähigkeit, unsere Begierden und Wünsche zu kontrollieren) anlehnt. Anders ausgedrückt: Würdig verhält sich, wer <<sparsam, enthaltsam, streng und nüchtern>> ist. 

\subsection{Würde reduziert auf Grundrechte}
Dieter Birnbacher reduziert die Würde auf vier elementaren Grundrechte. Gemäss ihm wird Würde verletzt, wenn auch nur eines dieser Grundrechte beeinträchtigt wird, egal ob alle anderen überkompensieren (Ein Grundrecht kann nicht durch andere aufgewogen werden). Diese Würde hat keine normative Bedeutung, d.h. sie taxiert lediglich die juridischen Rechte eines Menschen. 

\subsection{Kant's Würdeverständnis}
\begin{warningbox}
Würde an anderen Menschen, d.i. eines Werths, der keinen Preis hat, kein Äquivalent verstattet. Sie hat einen absoluten Wert.
\end{warningbox}

Kant interpretiert Würde als den moralischen Anspruch, in bestimmten Weisen nicht behandelt zu werden. Gemäss ihm ist sie etwas inhärentes, unreduzierbares, das geachtet werden muss (sowohl die eigene, wie auch die fremde) und das alle Wesen tragen, die \textit{autonom} sind. Wobei autonom bei Kant bedeutet, dass dieses Wesen sich selber das Gesetz gibt und ihm nicht einfach unterworfen ist. 

Entgegen des relativen Preises (beziffernbar, z.B. Marktpreis oder Affektionspreis) hat Würde einen \textit{absoluten} Preis. Das heisst, der Wert der Würde kann nicht verglichen werden. 

Da alle Menschen Würde haben, sind wir in der Pflicht, diese zu respektieren. Das heisst, wir dürfen Wesen mit Würde nicht bloss als Zweck verwenden (wir respektieren die eigenen Ziele des würdetragenden Wesens). Als Beispiel lässt sich hier das <<falsche Versprechen>> aufführen. Wer ein solches abgibt, der respektiert die Würde der anderen Person nicht. 



\paragraph{Grundfrage} Was ist alles unter Schädigung zu verstehen?

\paragraph{Antwort}

\end{document}