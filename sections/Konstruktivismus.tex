\documentclass[../main.tex]{subfiles}
\begin{document}
\begin{warningbox}
Der ethische Konstruktivismus besagt, dass Aussagen aufgrund eines vom Menschen kreierten idealen Verfahrens wahr oder falsch sind. 
\end{warningbox}

\subsection{Rawl's Gleicheits- und Differenzierungsprinzip (als Beispiel)}

\paragraph{These} Gerecht ist eine Gesellschaft, die ihre Güter nach dem Differenzierungsprinzip verteilen würden.

\paragraph{Argument} 
Das ideale Verfahren ist, gemäss Rawl, eines, dass soziale und wirtschaftliche Ungleichheiten so gestaltet, dass sie zu jedermanns Vorteil dienen und sie mit Positionen und Ämtern verbunden sind, die jedem offen stehen. Er meint, dass dies der Fall sei, wenn sich Menschen hinter dem "Schleier des Nichtwissens" einigen würden. Dieser Schleier verunmöglicht dem Entscheider, zu sehen, welche Position er in der Gesellschaft hat.

\end{document}