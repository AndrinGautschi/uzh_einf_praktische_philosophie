\documentclass[../main.tex]{subfiles}
\begin{document}
\begin{warningbox}
Die Deontologie, entgegen dem Konsequenzialismus, verbietet es, in gewissen Fällen das Beste zu tun, damit wir 
\begin{enumerate}[label=(\alph*)]
	\item Erlaubnisvorschlag I: unsere Interessen schützen können,
	\item (Erlaubnisvorschlag II) unsere Rechte wie z.B. Eigentumsrechte wahrnehmen können,
	\item Kant: andere nicht \textit{unzulässig instrumentalisieren}, also bloss als Mittel verwenden,
	\item die moralischen Rechte anderer nicht verletzten
	\item oder unseren eigen auferlegten Verpflichtung (z.B. Versprechen gegenüber anderen) wahren können
\end{enumerate}

\end{warningbox}

Während der Konsequenzialismus (\ref{konsequenzialismus}) die normativen Eigenschaften einer Handlung an seinen (wahrscheinlichen) Folgen misst, versucht die Deontologie die Handlungen aufgrund der Handlungen selber einzuschätzen. Sie erlaubt, nicht das Beste zu tun oder verbietet dies. 

\paragraph{Grundfrage} Sollen wir wirklich jeweils das tun, was den grössten Erwartungswert hat?

\paragraph{Antwort} Nein. Es gibt Fälle, die es uns erlauben, nicht das Beste zu tun und es gibt Handlungen, die intrinsisch schlecht sind und deswegen verboten sind. 

\subsection{Samuel Schefflers Erlaubsnisvorschlag 1}
In bestimmten Fällen ist es erlaubt, nicht das Beste zu tun. Dies ist der Fall, wenn die Beste Handlung unsere eigenen Interessen, Anliegen und Projekte gefährden würde. Demzufolge ist der persönlichen Perspektive besonderes Gewicht zuzukommen. 

\subsection{Erlaubnisvorschlag 2}
Rechte erlauben uns, nicht immer das Beste zu tun. Es kann also nicht falsch sein, seine Rechte zu benutzen, selbst wenn dies negative Konsequenzen für andere hat. Dieser Sachverhalt ist im Konsequenzialismus (\ref{konsequenzialismus}) nicht gegeben. 

\subsection{Kant's Verbot der besten Handlung}
Liegt einer Handlung eine \textit{unzulässige Instrumentalisierung} (jemand wird bloss als Mittel behandelt) vor, ist diese verboten. In Kant's \textit{kategorischem Imperativ} wird dies weiter verdeutlicht; Handle so, dass du alle involvierten jederzeit Zweck und Mittel sind, niemals nur Mittel. Als Begründung für diese Haltung wird die Würde des Menschen angeführt. Diese werde, sollte der Mensch als Mittel eingesetzt werden, zwangsweise verletzt, da dieser Mensch nicht dazu einwilligen kann und auch keine akzeptablen Gründe bestehen, warum er dies tun würde. Daraus folgt, dass wir unter einer \textit{unvollkommenen Pflicht} stehen, zum Glück anderer beizutragen und somit auch nicht immer verpflichtet sind, das Beste zu tun. 

\subsection{Verbot der besten Handlung aufgrund moralischer Rechte}
Handeln wir konsequenzialistisch am Besten und verletzten wir dabei die moralischen Rechte anderer, ist dies verboten. Das heisst, dass Menschen Rechte haben, die wir respektieren müssen, selbst dann, wenn deren Verletzung summiert bessere Folgen hätte. 

\subsection{Ross's Verbot der besten Handlung}
W.D.Ross (1877-1971) merkt an, dass es gewisse Pflichten gibt, die uns verbieten, das Beste zu tun. Dies sei zum Beispiel in Falle von Versprechen der Fall. Kein Konsequenzialist könne Versprechen abgeben.

\end{document}