\documentclass[../main.tex]{subfiles}
\begin{document}
 

\begin{warningbox}
Der moralische Realismus als Teil der Metaethik besagt, dass es eine unabhängige moralische Realität gibt, nach der Aussagen wahr oder falsch sind.
\end{warningbox}

\paragraph{These} Moralische Aussagen sind wahr/falsch, abhängig von einer äusseren moralischen Eigenschaft

\paragraph{Begründung} Supervenienz: Handlungen haben bestimmte nicht-moralische Eigenschaften inne (z.B. die Eigenschaft <<Schmerzen zufügend>>). Über diesen supervenieren die moralischen Eigenschaften (z.B. <<verwerflich>>). Für das spricht
\begin{itemize}
  \item unsere Praxis der Moral, also dass wir Moral nicht für eine Geschmackssache halten.
  \item Zusätzlich baut unsere Welt auf diesen Grundsätzen auf und man müsste zu viel ändern, wäre es nicht so. 
\end{itemize} 

\paragraph{Einwände des Realismus gegen Mackie's Irrtumstheorie} 
\begin{itemize}
  \item \textbf{Absonderlichkeit} \\ Wir tun Dinge, \textit{weil} wir sie für gut und richtig halten. Unsere Wünsche und Ziele entspringen aus unseren Vorstellungen des Guten und Richtigen.
  \item \textbf{Relativität} \\ Gesellschaftliche Unterschiede in den moralischen Systemen lassen sich auf unsere Irrtümer in der Auslegung zurückführen. Mögliche Quellen von Irrtümern sind Eigen- und Gruppeninteressen und Traditionen. Diese Irrtümern sind hartnäckiger, denn ihre Falsifizierung würde bedeuten, dass wir falsch gelebt haben. 
\end{itemize}

\paragraph{Einwände des Realismus gegen Konstruktivismus} Ein ideales Verfahren t sich nicht finden, denn unsere Beurteilung von den Ergebnissen dieses Verfahrens orientieren sich an geglaubten Fakten der Moral. Fakten, die dem Verfahren vorausliegen.


\end{document}