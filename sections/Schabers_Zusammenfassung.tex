\documentclass[../main.tex]{subfiles}
\begin{document}
 
\section{Block 1}
\begin{itemize}
  \item Es geht in der praktischen Philosophie darum zu bestimmen, was sein soll.
  \item Ethik hat die Moral zum Gegenstand.
  \item Deskriptive Aussagen wollen uns sagen, was ist. 
  \item Normative Aussagen wollen uns sagen, was sein soll.
  \item Mit evaluativen Aussagen werden Handlungen bewertet.
  \item Deontische Aussagen sagen, was geboten, verboten und erlaubt ist.
  \item A kann einen moralischen Grund haben, x zu tun, ohne dazu verpflichtet zu sein. Dann kann A fur die Unterlassung von x kein Vorwurf gemacht werden.
  \item A kann verpflichtet sein, x zu tun. Dann kann A fur die Unterlassung von x ein Vorwurf gemacht werden.
  \item Aus deskriptiven Aussagen lassen sich keine normativen Aussagen ableiten (das Humesche Gesetz/naturalistischer Fehlschluss).
\end{itemize}

\section{Block 2}
\begin{itemize}
	\item Nach Auffassung ethischer Nonkognitivist*innen sind moralische Aussagen weder wahr noch falsch, sondern Ausdruck von Gefuhlen und Einstellungen.
	\item Es gibt kein System richtiger moralischer Uberzeugungen (Ayer).
	\item Mackies Irrtumstheorie: Obwohl wir moralische Aussagen nicht als reine Expressionen verstehen, kann es nichts geben, was moralische Aussagen wahr macht.
	\item Die grundlegenden und hartnackigen moralischen Meinungsunterschiede sprechen nach Mackie gegen das Bestehen moralischer Tatsachen (Argument aus der Relativitat).
	\item Moralische Tatsachen waren nach Mackie absonderliche Dinge, da es Fakten waren, die uns motivieren und uns Grunde geben konnten (Argument aus der Absonderlichkeit).
\end{itemize}

\section{Block 3}
\begin{itemize}
	\item Nach Auffassung ethischer Konstruktivisten sind moralische Aussagen wahr, wenn sie aus einem von Menschen durchgefuhrten Verfahren hervorgehen.
	\item Nach Auffassung moralischer Realisten sind moralische Aussagen wahr oder falsch nach Massgabe einer von unseren Uberzeugungen und Wunschen unabhangigen moralischen Realitat.
	\item Das entspricht unserem Verstandnis des Moralischen: 
	\begin{itemize} 
		\item Es geht dabei nicht bloss um unsere Einstellungen;
		\item Wir konnen uns in moralischen Belangen irren.
	\end{itemize}
	\item Moralische Realisten weisen sowohl das Argument aus der Relativitat wie auch das Argument aus der Absonderlichkeit zuruck.
\end{itemize}

\section{Block 4}
\begin{itemize}
	\item Moralische Normen sind universalisierbar (1+2).
	\item Moralische Normen haben einen bestimmten Inhalt.
	\item Normen werden als moralische Normen verstanden, wenn man bestimmte Reaktionen auf deren Verletzung fur angemessen halt.
\end{itemize}

\section{Block 5}
\begin{itemize}
	\item Nach Auffassung von Konsequentialisten ist eine Handlung moralisch richtig, wenn sie das Gute optimal befordert und moralisch falsch, wenn sie das nicht tut.
	\item Deontologinnen meinen, dass es manchmal erlaubt ist, das Gute nicht zu befordern.
	\item Deontologinnen meinen, dass es manchmal moralisch falsch ist, das Gute zu befordern.
	\item Nach Kant ist das falsch, wenn wir andere dabei bloss als Mittel benutzen.
	\item Andere glauben, dass das falsch ist, wenn dabei Rechte von Menschen verletzt werden.
	\item Ross meint, dass das falsch ist, weil wir die Pflicht haben konnen - wie im Fall eines Versprechens -, nicht das Beste zu tun.
\end{itemize}

\section{Block 6}
\begin{itemize}
	\item Rechte sind Privilegien, Immunitaten, Powers und Anspruche.
Anspruchsrechte korrespondieren Pflichten. 
	\item Rechte korrespondieren Pflichten, die vom Rechtstrager kontrolliert werden konnen (die Willenstheorie). 
	\item Rechte korrespondieren Pflichten, die uns vorschreiben,  Interessen der Rechtstrager nicht zu verletzen und in bestimmten Fallen zu bedienen (die Interessentheorie der Rechte).  
\end{itemize}

\section{Block 7}
\begin{itemize}
	\item Komparativer Schadensbegriff: Nach einem komparativen Schadensbegriff hangt die Antwort auf die Frage, ob A durch Ereignis E geschadigt worden ist, davon ab, wie das Leben von A verlaufen ware, wenn E nicht eingetreten ware.  
	\item Nicht-komparativer Schadensbegriff: Ob A durch E geschadigt worden ist, hangt davon ab, ob As Leben durch E verschlechtert worden ist. 
\end{itemize}

\section{Block 8}
\begin{itemize}
	\item Eine Schadigung kann darin bestehen, dass man\begin{enumerate}[label=(\alph*)]	\item Eine Praferenz (Wunsch) durchkreuzt.
			\item Das subjektive Wohlergehen reduziert.
			\item Ein objektives Gut beeintrachtigt.
		\end{enumerate}
\end{itemize}

\section{Block 9}
\begin{itemize}
	\item Es gibt einen nicht-komparativen und einen komparativen Begriff der Schadigung. 
	\item Man schadet jemandem, wenn man a) seine Praferenzen durchkreuzt, b) sein Wohlergehen vermindert oder c) ein objektives Gut beeintrachtigt/zerstort. 
	\item Einer Person zu schaden, heisst nicht immer ihr ein Unrecht zuzufugen (freiwillige Einwilligung, Ausubung von Rechten, Notwehr, in Kauf nehmen, um Gutes zu realisieren).   
\end{itemize}


\section{Block 10}
\begin{itemize}
	\item B wird von A gezwungen, x zu tun, wenn\begin{enumerate}[label=(\alph*)]
			\item B glaubt, dass A seine Situation verschlechtern wurde, wenn sie x nicht tun wurde;
			\item B glaubt, dass A die Drohung wirklich wahr machen wurde, wenn sie x nicht tun wurde;
			\item B x tut, weil sie die Verschlechterung vermeiden will.
		\end{enumerate}
	\item 2 Begriffe von Zwang: \begin{enumerate}[label=(\alph*)]
			\item Nicht-moralischer Begriff: Wenn die Androhung eines erheblichen Ubels nicht zum normalen Verlauf der Dinge gehort.
			\item Moralischer Begriff: Wenn die Androhung eines erheblichen Ubels moralisch nicht in Ordnung ist.
		\end{enumerate}
\end{itemize}

\section{Block 11}
\begin{itemize}
	\item Ausbeutung stellt eine unfaire Transaktion dar.
	\item Unfair ist eine Transaktion, wenn eine Partei nicht das erhalt, was ihr zusteht. 
	\item Eine unfaire Transaktion kann beide Parteien besserstellen.
	\item In Ausbeutung kann man freiwillig einwilligen.
 	\item Offene Fragen:\begin{enumerate}[label=(\alph*)]
			\item Ist Ausbeutung, in die freiwillig eingewilligt wurde, moralisch erlaubt?
			\item Moralisch erlaubt, wenn die Einwilligung unter normalen Umstanden gegeben wird?
			\item Moralisch auch erlaubt, wenn die Einwilligung problematisch ist (in einer Notsituation gegeben wird)? 
		\end{enumerate}
\end{itemize}


\section{Block 12}
\begin{itemize}
	\item Autonomie: \begin{enumerate}[label=(\alph*)]
			\item Die Person i) weiss, was sie tut, ii) sie hat die Absicht, das, was sie im Begriff ist zu tun, zu tun, und iii) das, was sie tut, tut sie freiwillig (Beauchamp/Childress).
			\item Volitionen 2. Ordnung sind handlungsleitend (Frankfurt).
			\item Autorin des eigenen Lebens sein (Raz).     
		\end{enumerate}
	\item Anti-Paternalismus: Autonome Entscheidungen sind zu respektieren.
	\item Paternalismus: Man darf in bestimmten Fallen autonome Entscheidungen, die nicht im Interesse der Person sind, missachten und die Betroffenen zu ihrem Guten zwingen oder verfuhren.
	\item 2 Formen des Paternalismus:\begin{enumerate}
			\item Zwangspaternalismus
			\item Libertarer Paternalismus
		\end{enumerate}
	\item 2 Ziele des Paternalismus:\begin{enumerate}
			\item Die eigenen Ziele des Betroffenen zu verwirklichen
			\item Objektiv gute Ziele zu befordern
		\end{enumerate}
\end{itemize}


\section{Block 13}
\begin{itemize}
	\item Mit Einwilligungen setzen wir Rechte ausser Kraft und entlassen damit andere aus den korrespondierenden Pflichten, die sie uns gegenuber haben.
	\item Um normativ wirksam zu sein, mussen Einwilligungen informiert und freiwillig sein und von einer einwilligungsfahigen Person gegeben werden. 
	\item Wenn es unverausserliche Rechte gibt, konnen auch gultige Einwilligungen normativ unwirksam sein. 
\end{itemize}


\section{Block 14}
\begin{itemize}
	\item Wurde wird verstanden als \begin{enumerate}[label=(\alph*)]
			\item Ensemble von Grundrechten 
			\item als absoluter Wert
		\end{enumerate}
\end{itemize}
 
\end{document}