\documentclass[../main.tex]{subfiles}
\begin{document}

\begin{warningbox}
Der ethische Nonkognitivismus als Teil der Metaethik beantwortet die Frage, ob es Moral gibt/geben kann, damit, dass bestimmte Fragen prinzipiell nicht kognitiv, d.h. durch Erkenntnis mit wissenschaftlichen Mitteln, ermittelt werden können. Demzufolge gibt es keine Moral.
\end{warningbox}

\subsection{Ayers Argument}
\begin{warningbox}
Gemäss Alfred Ayer gibt es, aufgrund unmöglicher empirischer Nachweisbarkeit, keine Moral, sondern nur expressive Aussagen.
\end{warningbox}
\paragraph{These} Normative Aussagen sind weder wahr noch falsch. 
\paragraph{Argumentation}
Normative Aussagen (gut, schlecht, richtig, recht) sind lediglich expressive Aussagen, sie sagen also nur etwas über das emotionale Befinden des Sprechers aus. Denn wahr ist, gemäss Ayer, nur, was empirisch (Verifikationsprinzip) belegt oder widerlegt werden kann. Alles andere seien emotionale Offenbarungen. Demzufolge lässt sich die Aussage \textit{X ist schlecht} durch ein negativ betontes \textit{Buuuh!} ersetzen, ohne dessen Aussagengehalt zu verändern.


\paragraph{Moores Einwand}
Hätten moralische Urteile nur expressive Bedeutung, wären wir uns darüber nicht uneinig. D.h. moralische Urteile haben keine bloss expressive Bedeutung. 

\subsection{Mackies Irrtumstheorie}
\begin{warningbox}
Gemäss John Mackie kann es, obwohl wir moralische Aussagen nicht als reine Expressionen verstehen, aus Gründen der Relativität (keine Konvergenz von Werten in den verschiedenen moralischen Systemen) und der Absonderlichkeit (es gibt nur Wünsche, die für uns definieren, was gut/schlecht ist) nichts geben, was moralische Aussagen wahr macht. Demzufolge sind alle moralischen Aussagen falsch, jedoch immer noch dem Kognitivismus unterzuordnen.
\end{warningbox}

\paragraph{These} Normative Aussagen sind wahrheitsfähig, aber ausnahmslos falsch.
\paragraph{Argumentation}
Wir benutzen moralische Aussagen nicht bloss als emotionale Offenbarungen, sondern schaffen einen Bezug zur Objektivität indem wir davon ausgehen, dass Moral \emph{ist}. So sagen wir z.B., dass stehlen schlecht \emph{ist}, nicht wir finden stehlen schlecht. Gemäss Mackie kann etwas aber nur gut/schlecht sein, wenn es eine solche Eigenschaft hätte. Dies sei wegen folgend Gründen unmöglich:
\begin{itemize}
  \item \textbf{Relativität} \\
In verschiedenen moralischen Systemen gibt es, im Vergleich zu deskriptiven Aussagen, hartnäckige(re) und grundlegende(re) Meinungsverschiedenheiten. Selbst wenn man davon ausgehe, dass Menschen falsch liegen können, seien die Unterschiede zu gross um sie zu erklären. Denn diese Unterschiede enstehen durch das Umfeld, in dem wir aufwachsen. Gäbe es also eine Moral, so Mackie, hätten die Ansichten konvergieren sollen.
  \item \textbf{Absonderlichkeit} \\
Gäbe es Moral, gäbe es Fakten, die uns anweisen und zum Handeln motivieren. Dies sei aber nicht der Fall (und wäre absonderlich). Nur Wünsche/Ziele weisen und motivieren uns. Daraus folgt auch, dass alles, was unseren Zielerreichung ermöglicht, gut ist und alles, was diese verunmöglicht/erschwert schlecht ist. Es gibt also, so Mackie, kein gut oder schlecht, sondern nur Dinge, die wir haben oder tun möchten. 
\end{itemize}
\paragraph{Warum sind alle moralischen Aussagen falsch?} Sagen wir "Dieses Blatt ist rot" so drücken wir implizit aus, dass das Blatt die Eigenschaft rot hat. Dies ist jedoch falsch, denn das Blatt hat (angenommen es gäbe keine Farben) diese Eigenschaft nicht. 

\end{document}