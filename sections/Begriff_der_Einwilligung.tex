\documentclass[../main.tex]{subfiles}
\begin{document}

Eine Einwilligung bezieht sich auf die Handlung einer anderen Person und setzt voraus, dass die einwilligende Person über das Recht der Handlung verfügt. Dabei gibt es verschiedene Auffassungen, ab wann eine Einwilligung ausgedrückt wurde;
\begin{itemize}
	\item Mentale Zustände: Der Wunsch einer anderen Person eine Erlaubnis zu erteilen
	\item Mentaler Akt: Entscheidung, einer anderen Person eine Erlaubnis zu erteilen
	\item Kommunikativer Art: Einer anderen Person \textit{mitteilen}, dass man ihr eine Erlaubnis erteilen will
\end{itemize} 


\subsection{Normative Funktion der Einwilligung}
Die Einwilligung hat eine normative Funktion. Sie (1) dient der Erlaubnis, dass das Gegenüber eine bestimmte Handlung ausführen darf und (2) definiert, dass der einwilligenden Person kein Unrecht getan wird. Dies geschieht nach dem Einwilligungsschema (Unerlaub --> Einwilligiung --> erlaubt). Das bedeutet, dass die Einwilligung eine Ausübung des Rechts ist (ausser Kraft setzten der Pflicht eines anderen). Als Grundlage der Einwilligung gilt der Wille der einwilligenden Person (normativer Wille, ein Recht ausser Kraft zu setzen). Dies ist jedoch nicht zu verwechseln mit dem Wunsch, dass etwas geschieht. Eine Einwilligung ist also eine Entlassung eines anderen aus einer Pflicht mir gegenüber (zurückziehbar). Dem gegenüber ist das Versprechen, bei dem wir uns einer anderen Person gegenüber in die Pflicht stellen (nicht zurückziehbar). 

\subsection{Gültigkeitsbedingungen von Einwilligungen}
Damit eine Einwilligung gültig ist, muss sie folgende Bedingungen erfüllt:
\begin{enumerate}[label=(\alph*)]
	\item Man muss vollständig informiert sein, das heisst, dass die einwilligende Person vollständig über die (möglichen) Konsequenzen seiner Einwilligung informiert ist. Täuschungen und Manipulationen machen die Einwilligung also ungültig. 
	\item Die Entscheidung zur Einwilligung erfolgt freiwillig. Die einwilligende Person steht also nicht unter innerem oder äusserem Zwang. 
	\item Die Entscheidung wird von einem einwilligungsfähigem Wesen gegeben. Das heisst, die einwilligende Person versteht die normativen Konsequenzen seiner Handlung und ist in der Lage, die eigenen Interessen wahrzunehmen. 
	\item Es handelt sich um ein veräusserlichbares Recht. 
\end{enumerate}

\subsection{Moralische Grenzen der Einwilligung}
Eine Einwilligung ist genau dann wirksam, wenn sie ein Recht betreffen, das ausser Kraft gesetzt werden kann. Das heisst, dass, selbst bei Einwilligung, ein unveräusserliches Recht nicht ausser Kraft gesetzt wird und die Einwilligung somit die normativen Situation der Person nicht verändert. Man spricht hierbei von einer \textit{nicht hinreichenden} Bedingung und einer \textit{moralischen Nicht-Wirksamkeit}. 

Eine Einwilligung kann gültig, aber nicht wirksam sein. Das ist der Fall, wenn die Einwilligung freiwillig erfolgt, aber nicht, weil es sich um ein unveräusserliches Recht handelt, wirksam werden kann. 

\end{document}