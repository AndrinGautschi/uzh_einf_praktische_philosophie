\documentclass[../main.tex]{subfiles}
\begin{document}
\begin{warningbox}
Zwang ist, wenn B von A zu x gezwungen wird oder wenn A B droht und (gemäss Nozick 1997)
\begin{itemize}
	\item B glaubt, dass A seine Situation in unzumutbarer Weise verschlechtern würde, wenn sie x nicht tun würde;
	\item B glaubt, dass A die Drohung wirklich wahr machen  würde;
	\item B tut x, weil sie die Verschlechterung vermeiden will.
	\item (umstritten!) A nicht berechtigt ist, die Drohung auszusprechen
\end{itemize}
\end{warningbox}


Jemanden zwingen ist moralisch falsch und ist für Opfer meist eine begründete Entschuldigung, bestimmt gehandelt zu haben. 

\subsection{Arten des Zwangs}
\begin{enumerate}
	\item Physischer Zwang: Man zwingt jemanden physisch, etwas zu tun oder zu unterlassen, was diese Person ansonsten nicht getan hätte.
	\item Drohung: Durch ankündigen von negativen Konsequenzen wird jemand zu etwas gebracht, das er sonst nicht tun würde. Dies, weil Drohungen unsere Handlungsgründe verändern. Drohungen sind aber nicht mit Warnungen zu verwechseln; Eine Warnung ist ein Hinweis auf einen schon bestehenden Handlungsgrund, um negative Konsequenzen zu vermeiden. 
\end{enumerate}

\subsection{Nicht-moralischer Begriff von Zwang}
Beim nicht-moralischen Begriff (nmB) wird Zwang als etwas interpretiert, das nicht zum normalen Verlauf der Dinge gehört und durch die Handlung des Zwingenden zur Realität wird. So kann man nicht von Zwang sprechen, wenn die Kassiererin beim Nichtzahlen die Ausgabe der Wahre verwehrt, jedoch kann von Zwang gesprochen werden, wenn ich diese Kassiererin mit einer Waffe (und damit unter Schaffung einer neuen Konsequenz) zur Herausgabe zwinge. 

\subsection{Moralischer Begriff von Zwang}
Beim moralischen Begriff (mB) wird Zwang als etwas interpretiert, wozu der Zwingende moralisch nicht berechtig ist es zu tun. So ist, selbst wenn dies dem normalen Verlauf der Dinge in einer Gesellschaft entspricht, es falsch, einen Sklaven zu halten und zur Arbeit zu motivieren, weil es moralisch falsch ist, Sklaven zu halten. Jedoch ist es kein Zwang, wenn der Lehrer den Schüler unter Androhung von negativen Konsequenzen zur Hausaufgabe zwingt, denn der Lehrer ist dazu berechtigt.

\subsection{Angebote als Zwang}
Es stellt sich die Frage, ob Angebote auch Zwänge sind. Dies ist jedoch nur dann der Fall, wenn es sich um ein \textit{illegitimes Angebot} handelt. Also ein Angebot, zu welchem der Anbietende nicht berechtigt ist, z.B. wenn er etwas anbietet, dass er sowieso anbieten muss. Auch ist ein Angebot Zwang, wenn die Ablehnung des Angebots für die betroffene Person mit sehr hohen Kosten verbunden ist (z.B. Hungertod). Es handelt sich aber nicht um Zwang, wenn keine Rechte der betroffenen Person verletzt werden. 

 \subsection{Zwang und Moral}
 Zwang ist \textit{pro tanto} (teilweise) moralisch falsch. Es ist also dann moralisch falsch, wenn Rechte der betroffenen Person verletzt werden und es nicht gewichtigere moralische Gründe gibt, dies zu rechtfertigen. Solange jedoch keine Rechte verletzt werden, ist Zwang unproblematisch. 
 
 \subsubsection{Gerechtfertigter Zwang}
 \begin{enumerate}
 	\item Zwang ohne Rechtsverletzung: Wie oben erwähnt, ist Zwang, ohne dass Rechte verletzt werden, im Grunde unproblematisch
 	\item Verhinderung von moralischem Unrecht: Zwang ist erlaubt/geboten, wenn damit moralisches Unrecht verhindert werden kann. Dies ist im Falle von Verhinderung von Mord, Vergewaltigung, Folter, etc. der Fall.
 	\item Zwang als Mittel der Gerechtigkeit: Auch \textit{Strafgerechtigkeit}. Strafen haben immer das Element von Zwang. Dieses ist gerechtfertigt, wenn es der Gerechtigkeit dient.
 	\item Paternalistischer Zwang: Dies kann erlaubt/gerechtfertigt sein, wenn man damit jemanden vor Selbstschädigung schützt
 \end{enumerate}

\end{document}