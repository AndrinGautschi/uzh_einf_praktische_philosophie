\documentclass[../main.tex]{subfiles}
\begin{document}

\begin{warningbox}
B wird von A ausgebeutet, gdw. die Transaktion, die zwischen A und B stattfindet, unfair ist, d.h. B nicht bekommt, was ihr zusteht. 
\end{warningbox}

\subsection{Marx' Begriff von Ausbeutung}
Der Begriff der Ausbeutung wurde von Karl Marx geprägt, der diese im Tausch zwischen Arbeitskraft und Lohn sah. Weil der Arbeiter zu Lohnarbeit gezwungen ist und dabei Güter produziert, die vom Kapitalist auf den Märkten für mehr verkauft werden, als der Kapitalist dem Arbeiter zahlt, handelt es sich um Ausbeutung. Oder in anderen Worten: Der gezwungene Arbeiter erhält nicht den Mehrwert, den er für den Kapitalisten erzeugt. Marx stützt sich dabei auf die Annahmen, dass die Arbeit den Wert der Güter produziert, dass der produzierte Wert ein objektiver Wert ist (durchschnittlich für die Produktion des Gutes benötigte Arbeitszeit) und dass der Kapitalist die Produktionsmittel besitzt. Marx leitet daraus die Forderung ab, dass das Privateigentum an den Produktionsmitteln abzuschaffen ist.   

\subsection{Unfaire Transaktionen}
Gemäss A. Wertheimer gibt es folgende Arten der unfairen Transaktion:
\begin{enumerate}
	\item Schädliche Ausbeutung: Eine Partei wird durch eine Transaktion geschadet, während die andere Partei Vorteile daraus zieht. 
	\item Wechselseitig vorteilhafte Ausbeutung: A und B profitieren von der Transaktion, die Transaktion ist jedoch unfair (einer mehr als der andere). 
\end{enumerate}

\subsection{Zulässige Ausbeutung}
Wann handelt es sich um zulässige Ausbeutung? Wenn freiwillig Eingewilligt wurde? Ob eine Ausbeutung zulässig ist, ist abhängig von der Qualität der Einwilligung. Handelt es sich um eine \textit{echte Einwilligung}, also eine, die unter normalen Umständen gegeben wird, ist eine Ausbeutung zulässig. Unzulässig ist sie, wenn es sich um eine \textit{problematische Einwilligung} handelt. Also einer, die in einer Notsituation gegeben wird. 
\subsubsection{Ist Ausbeutung mit problematischer Einwilligung falsch?}
Gemäss A. Wertheimer ist dies abhängig davon, wie man gewisse moralische Gesichtspunkte gewichtet. So kann zum Beispiel eine problematische Einwilligung zur Ausbeutung moralisch zulässig sein, wenn sie zum Vorteil beider Betroffenen ist und wenn es für den Ausgebeuteten rational ist, sich auf die Transaktion einzulassen.

\end{document}