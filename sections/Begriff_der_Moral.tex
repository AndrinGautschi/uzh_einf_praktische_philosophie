\documentclass[../main.tex]{subfiles}
\begin{document}

\begin{warningbox}
Moral bezeichnet alle als wichtig und richtig anerkannte Normen und Ideale des guten und richtigen Sich-verhaltens. 
\end{warningbox}

\paragraph{Problem} Was sind moralische Normen? Wie lassen sich Aussagen als moralische Aussagen klassifizieren?

\subsection{Allgemeine Defintion}
<<Moral ... bezeichnet summarisch alle von einem Menschen oder Gesellschaft als richtig und wichtig anerkannten Normen und Ideale des guten und richtigen Sich-Verhaltens ...>> (Handbuch Ethik 2006: 426)

\subsection{Moral als universalisierbare Aussagen}
\begin{warningbox}	
(Gemäss R.M. Hares) Eine moralische Aussage ist gegeben, wenn sie \textit{universalisierbar} ist, d.h. sie ist unparteilich und auf alle anwendbar.
\end{warningbox}

\subsubsection{Universalisierbarkeit 1}
Ein moralisches Sollen ist ein Sollen, das Geltung besitzt, immer wenn Menschen bestimmte Eigenschaften haben, wie z.B. \textit{Alle, welche Eigenschaften a, b, c ... haben, sollten x tun}

\subsubsection{Universalisierbarkeit 2}
Ein moralische Sollen ist ein Sollen, wenn es sich daraus ergibt, dass man sich in alle von der Handlung Betroffenen versetzt und zum Schluss kommt, dass man die Handlung ausführen soll.

\subsection{Ein inhaltlicher Begriff der Moral}
\begin{warningbox}
(Gemäss Philipa Foot) Moral hat einen bestimmten Inhalt, welcher durch normative Theorien gegeben ist, die der Förderung von gesellschaftlichem Wohl oder den Achtung von Werten dienen. 
\end{warningbox}

Bei moralischen Aussagen geht es um die Beförderung des Wohls oder der Achtung von Werten. Um den Inhalt zu kennen, benötigen wir Wissen über die Moral.

\paragraph{Problem} Ayn Rand schlug die Idee einer egoistischen Moral vor und wirft damit die Frage auf, ob wir hier noch von Moral reden oder ob wir über etwas ganz anderes sprechen. So sprechen wir von keinem etwas anderem, wenn wir wir uns am inhaltlichen Begriff der Moral orientieren, nach dem Moral die Gesellschaft fördern soll, nicht jedoch, wenn wir einen formalen Begriff der Moral als Orientierung verwenden. 

\subsection{Ein formaler Begriff der Moral}
\begin{warningbox}
Moralische Normen sind diejenigen Normen, über deren Verletzung wir uns empören und für deren Verletzung wir uns (1) schuldig fühlen, (2) uns entschuldigen und (3) Reue zeigen sollten.
\end{warningbox}

Moralische Normen erkennt man an den Reaktionen auf ein Fehlverhalten, die man für angemessen hält. Verhalten wir uns moralisch falsch, so sind angemessene Reaktionen (1) ein schlechtes Gewissen, (2) den Drang uns zu entschuldigen und (3) Reue. Dies ist nicht der Fall bei anderen Arten des Fehlverhaltens, so zum Beispiel wenn wir unklug handeln oder Anstandsregeln verletzen. 

\paragraph{Problem} Diese Norm sagt nicht aus, über was wir uns empören sollten. 

\end{document}