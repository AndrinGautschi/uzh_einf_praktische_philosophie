\documentclass[../main.tex]{subfiles}
\begin{document}
\begin{warningbox}
	Person A wird durch ein Ereignis geschädigt, gdw. es A besser ginge, wäre das Ereignis nicht geschehen. Es wird unterschieden, (nicht-komparativ) ob eine Schädigung unmittelbar Auswirkungen hat oder (komperativ) das Leben als gesamtes schlechter macht. Auch wichtig ist, ob es sich um eine Schädigung handelt; gemäss der \textit{Präferentheorie} dann, wenn Wünsche durchkreuzt werden, gemäss der Theorie des \textit{subjektiven Wohleregehens} dann, wenn sie sich schlechter fühlt oder gemäss der \textit{objektiven Liste Theorie} dann, wenn ein objektiv gutes Gut beeinträchtig wird. 
\end{warningbox}

\paragraph{Grundfrage} Was ist alles unter Schädigung zu verstehen?

\subsection{Ein nicht-komperativer Begriff}
Gemäss dem nicht-komperativen Begriff von Schädigung ist etwas schädigend, gdw. ein Ereignis in der Person Schmerz, mentales oder physisches Unwohlsein, Krankheit, Fehlbildung, Behinderung oder Tod auslöst. Dies ungeachtet dem möglichen Verlauf der Dinge, wäre das Ereignis nicht eingetreten. 

\paragraph{Schwierigkeiten} zeigen sich mit dem nicht-komperativen Begriff in Szenarien wie einem schmerzhaften medizinischen Eingriff. Auch stellt sich die Frage, ob, unterlässt man die Hilfestellung für eine Person mit einem schlechtes Leben, es sich um eine geringe Schädigung handelt, sollte die Person daraufhin sterben. 

\subsection{Ein komperativer Begriff}
Gemäss dem komperativen Begriff von Schädigung wird eine Person geschädigt, gdw. das Leben, das die Person mit dem Ereignis führt schlechter ist, als wäre das Ereignis nicht eingetreten. 

\paragraph{Schwierigkeiten} zeigen sich mit dem komperativen Begriff, wenn es um Abtreibung und Verhalten in der Vergangenheit geht; namendlich das \textit{Non-identity Problem}, bei welchem Abtreibung per se als Schädigung wahrgenommen wird, weil die Person durch eine (fehlende) Handlung gar nicht existieren würde und somit ihr, in allen Fällen lebbares Leben, nicht führen könnte. 

\subsection{Varianten von Schädigung}
\begin{enumerate}[label=(\alph*)]
	\item Jemandem nicht helfen: Einem nicht-komperativer Begriff (NKB) zufolge liegt keine Schädigung vor, das lediglich nicht verbessert wurde. Einem komperativen Begriff (KB) zufolge liegt eine Schädigung vor, da das Leben besser gelaufen wäre. 
	\item Verbesserung aktiv verhindern: NKB zufolge wird das Leben nicht verschlechtert, KB zufolge schon.
\end{enumerate}

\subsection{Was schlecht ist}
Um zu erfassen, was als Schädigung zählt, muss definiert sein, was schlecht ist. Dies könnte der Fall sein, wenn Präferenzen durchkreuzt werden, wenn das subjektive Wohlergehen gemindert oder wenn ein objektives Gut beeinträchtig wird.

\paragraph{Präferenztheorie:} Eine Person erleidet Schaden gdw. ein Wunsch dieser Person durch ein Ereignis durchkreuzt wurde.
\paragraph{Theorie des subjektiven Wohlergehens:} Eine Person erleidet Schaden gdw. sie sich nach einem Ereignis schlechter fühlt.
\paragraph{Objektive Liste Theorie:} Eine Person erleidet Schaden gdw. etwas durch ein Ergeignis beeinträchtigt wurde, das objektiv gut für diese Person ist. 

\subsection{Schädigung und Unrecht}
Nicht immer sind Schädigungen unrecht. Ebenfalls ist nicht alles Unrecht eine Schädigung (\textit{harmless wrongdoing}). 

\subsubsection{Erlaubte Schädigungen}
\begin{warningbox}
Keine Schädigung liegt vor, wenn wir diese nicht intendieren, wir zu ihnen (falls veräusserlichbares Recht) einwilligen, sie als Produkt der Wahrnehmung unserer Rechte entstehen, sie aus Notwehr resultieren oder als unintendiertes Nebenprodukt von einer guten Handlung erfolgen. 
\end{warningbox}

\begin{enumerate}[label=(\alph*)]
	\item Gewisse Schädigungen, die als nicht-intendierte Nebenfolge unseres Handelns entstehen und für die freiwillig eingewilligt (setzt veräusserlichbare Rechte voraus!) wurde wie z.B. Schädigung bei sportlichen Aktivitäten oder sadomasochistischen Praktiken.  
	\item Schädigung als Nebenfolge der Ausübung von Rechten wie z.B. das Beenden einer Partnerschaft, das Verdrängen eines Konkurrenten oder jemanden zu einer Freiheitsstrafe verurteilen.
	\item Schädigung aus Notwehr (Schädigung als einziges Mittel, sein Leben zu schützen)
	\item In Kauf genommene Schädigung um einen Zweck zu realisieren, der anders nicht realisierbar ist wie z.B. beim Konsequenzialismus, dass das Gute das Schlechte überwiegen muss oder im Falle des \textit{Prinzipes der Doppelwirkung} (PDW), bei welchem Schädigungen in Kauf (nicht beliebig viel und nicht beabsichtigt!) werden dürfen, wenn gutes im intendiert wird.  
\end{enumerate}

\end{document}