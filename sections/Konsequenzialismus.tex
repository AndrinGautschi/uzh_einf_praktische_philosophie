\documentclass[../main.tex]{subfiles}
\begin{document}
\begin{warningbox}
Der Konsequenzialismus will Gutes maximieren und Schlechtes minimieren. Dabei bestimmt er den normativen Wert anhand der Konsequenzen einer Handlung. Der Konsequenzialismus formuliert ein Prinzip des moralischen Handelns (<<Richtig handelt man, wenn man das tut, was die bestmöglichen Folgen hat>>).
\end{warningbox}

\paragraph{Grundfrage} Was ist die beste Handlung und wie bestimme ich sie?

\paragraph{Antwort} Das Beste orientiert sich am Ergebnis der Handlung und ist das, was für die von einer Handlung betroffenen insgesamt betrachtet die grössten Vorteile hat.

\subsection{Was ist gut?}
Gut ist, 
\begin{itemize}
  \item was Freude bereitet (Hedonismus)
  \item was Interessen dient (Interessenstheorie)
  \item ...
\end{itemize}
Wir haben also Gründe, gutes zu tun und zu unterlassen, was schlecht ist (ausser es dient dem Guten).

\subsection{Was ist das Beste?}
Das Beste setzt sich zusammen aus dessen rationalen Anwendung (das Beste für \textit{mich}) und dessen moralischen und unparteiischen Anwendung (das Beste für \textit{alle Betroffenen}). Daraus ergibt sich: Das Beste ist, was für alle Betroffenen insgesamt das Beste ist. 

\subsection{Maximaler Erwartungswert}
\begin{infobox}
Erwartungswert = Wert der Folgen * Eintrittswahrscheinlichkeit
\end{infobox}
Der \textit{maximale Erwartungswert} entsteht, wenn wir den Resultaten unserer Handlung Eintrittswahrscheinlichkeiten zuweisen, diese mit deren Nutzen multiplizieren und dann, gemäss der \textit{konsequentialistischen Grundnorm} den Höchsten auswählen. Beim Evaluieren der Eintrittswahrscheinlichkeit spielen statistische Grössen (statistische Eintrittswahrscheinlichkeit) und Überzeugungen (subjektive Gewissheit, dass etwas eintritt) eine Rolle. 

\subsection{Subjektiver vs. objektiver Konsequenzialismus}
\paragraph{Subjektiver Konsequenzialismus} Der subjektive Konsequenzialismus betrachtet den maximalen Erwartungswert zum Zeitpunkt des Handelns. Der Natur von Wahrscheinlichkeiten zufolge kann somit eine Handlung aber negativ (nicht das Beste) ausfallen
\paragraph{Objektiver Konsequenzialismus} Der objektive Konsequenzialismus betrachtet die eingetretenen Folgen des Handelns. Gut gehandelt hat, wer die besten Folgen erzielt. Somit ist eine Beurteilung nur im Nachhinein möglich und es werden keine Handlungsanleitungen gegeben. Eine Person kann gemäss dem OK falsch handeln, ohne dass man ihm Vorwürfe machen kann (<<blameless wrongdoing>>).

\subsection{Handlungs- vs. Regelkonsequenzialismus}
\paragraph{Handlungskonsequenzialismus} Beim Handlungskonsequenzialismus wird jede Handlung einzeln abgewägt. Dies ist jedoch im Alltag kaum umsetzbar.
\paragraph{Regelkonsequenzialismus} Beim Regelkonsequenzialismus werden Richtlinien (Regeln) aufgestellt, nach denen man handeln soll. Dies ist im Alltag umsetzbar. 

\end{document}