\documentclass[../main.tex]{subfiles}
\begin{document}
\begin{warningbox}
Autonom handelt man, wenn man 
\begin{enumerate}
	\item Beauchamp: (1) weiss was man tut, (2) mit Absicht handelt und (3) freiwillig ist
	\item Frankfurt: nach Volitionen 2. Ordnung handelt, also den eigenen Wünschen nicht nachgeben muss. 
\end{enumerate}
\end{warningbox}

Die Autonomie von Personen ist von der breiten Masse als eines der grundlegensten Prinzipien der Moral angesehen. Sie ist alltäglicher Gegenstand der Politik (Wie hat ein Staat die Autonomie seiner Bürger zu achten?) und anderer Bereiche wie der Medizin, etc. 

\paragraph{Grundfrage} Was ist unter Autonomie zu verstehen?

\paragraph{Antwort} Autonom zu handeln bedeutet, das man das tut, was man selber tun will, wenn also der eigene Wille handlungsleitend ist. 

\subsection{Freiheit}
Eine Entscheidung ist \textit{frei}, wenn sie nicht das Resultat von Zwang ist. Oder anders ausgedrückt: Handelt man unter Zwang, ist die Entscheidung nicht frei. Frei zu handeln bedeutet nicht autonom zu handeln!

\subsection{Freiwilligkeit} 
Freiwilligkeit ist das, was man die Abwesenheit von \textit{äusserem und innerem} Zwang nennt. Freiwillig ist nicht gleich autonom! Aber was bedeutet Freiwilligkeit genau? Dafür gibt es folgende zwei Vorschläge
\begin{enumerate}[label=(\alph*)]
	\item Was man tut, ist nicht freiwillig, wenn das, was man tut, aus einer Aktivität anderer hervorgeht, zu der sie nicht berechtigt sind (Nozick)
	\item Was man tut, ist nicht freiwillig, wenn man etwas nur deshalb tut, weil alle Alternativen inakzeptabel sind (Olsaretti). Dies ist im \textit{prudentiellen} Sinn zu verstehen; Es ist nicht akzeptabel im Blick auf das eigene Wohlergehen. Es ist jedoch hervorzuheben, dass eine Handlung freiwillig ist, sobald ein Grund dafür spricht, selbst dann, wenn alle Alternativen inakzeptabel sind!
\end{enumerate} 

\subsection{T. Beauchamps Vorschlag der Autonomie}\label{beauchampsautonomie}
Eine Person handelt autonom gdw:
\begin{enumerate}[label=(\alph*)]
	\item sie weiss, was sie tut
	\item sie die Absicht hat, das, was sie im Begriff ist zu tun, zu tun
	\item das, was sie tut, freiwillig tut.
\end{enumerate}

\subsection{H. Frankfurts Vorschlag der Autonomie}
\begin{warningbox}
Frankfurt: Autonom handelt man, wenn man diese Wünsche verfolgt, die sich auf meine Wünsche beziehen. 	
\end{warningbox}

Frankfurt unterscheidet in seiner Definition der Autonomie Wünsche und Volitionen (bewusste und zielgerichtete Umsetzung des eigenen Willen) erster und zweiter Ordnung. Wünsche erster Ordnung zu haben bedeutet, Wünsche zu haben, die man umsetzen möchte. Wünsche zweiter Ordnung sind Wünsche, die sich auf die erster Ordnung beziehen (ich will diesen Wunsch [nicht] haben). Die Volitionen beziehen sich auf diese Wünsche erster und zweiter Ordnung und besagen im Falle der zweiten Volition: Ich möchte den Wunsch erster Ordnung nicht wirksam werden lassen. Für Frankfurt ist Handeln genau dann autonom, wenn Volitionen zweiter Ordnung das Handeln leiten. Das bedeutet, dass, egal welche Wünsche ich habe, solche die ich nicht haben möchte oder solche, die ich haben möchte, lasse ich mich von diesen Wünschen nicht beherrschen, sondern kann selektiv diese Wünsche verfolgen, die ich als gut empfinde.
% TODO verstehe Frankfurts Vorschlag nicht ganz

\subsection{Joseph Ratz' Autonomie als Ideal}\label{ratzautonomieideal}
Gemäss Ratz bedeutet Autonomie, dass jemand der Autor seines eigenen Lebens ist. Das bedeutet, dass man unabhängig von anderen ist, dass das eigene Handeln auf der eigenen Reflexion beruht und nicht auf der Vorstellung anderer, Tradition oder Gemeinschaft und man ist autonom, wenn man die eigene Lebensform wählen kann. 

\subsection{Angewandte Autonomie}
Welches dieser Begriffe ist nun relevant? Die Anwendung bestimmter Definitionen ist abhängig vom praktischen Kontext. 

\subsection{Respekt vor Autonomie}
\begin{warningbox}
Autonome Entscheidungen sollten geachtet werden. Was autonom ist, bestimmt der Kontext. Und autonome Entscheidungen, sollten nicht \textit{immer} geachtet werden. 
\end{warningbox}

Im Begriff der \textit{Autonomie anderer} gibt es zwei Auffassungen; Die \textit{negativ bestimmte}, nach deren wir andere nicht daran hindern, das zu tun, was sie tun wollen und die \textit{positiv bestimmte}, nach deren wir anderen helfen, das zu tun, was sie tun wollen. Die Unterscheidung beruht also auf der Frage, ob wir die Autonomie anderer allein ihr Ding ist, oder ob wir sie bei der Erfüllung dieser unterstützen sollten. 

Die Standardposition ist folgende: Wir können die Autonomie anderer respektieren und entsprechend handeln, sind dazu jedoch nicht verpflichtet. Wie dies genau zu interpretieren ist, erläutern die folgenden Kapitel. 

\subsubsection{Suzanne Uniacke's Handlungsansatz}
Gemäss Uniacke verpflichtet uns der autonome Wille anderer nicht, verändert jedoch (solange nicht unmoralische, respektlose oder unhöfliche Ding gewollt sind) unsere Gründe zum Handeln. Dies ist aber nur in bestimmten Beziehungen wie der Freundschaft oder Dienstleistungen der Fall!

\subsubsection{Beauchamp's Autonomieanwendung}
Gemäss Beauchamp ist die Autonomieanwendung (gegenüber anderen) kontextabhängig. So ist er der Überzeugung, dass autonome Entscheidungen zu respektieren sind, unabhängig dessen, ob diese abgeleitet von anderen Dingen sind, solange die in \ref{beauchampsautonomie} aufgeführten Punkte eingehalten werden. Betrachtet man jedoch Gebiete wie die Erziehung, so ist für ihn die Interpretation erstrebenswerter, die Joseph Ratz definiert (\ref{ratzautonomieideal}).

\subsubsection{Sollten autonome Entscheidungen immer geachtet werden?}
Nicht, wenn man dabei anderen Unrecht tut (\textit{moralische Grenzen}) oder wenn die Handlung zuwider den eigenen Interessen ist (\textit{nicht-moralische Grenzen}).

\subsubsection{Achtung vor Autonomie zuwider der eigenen Interessen}
Wenn autonome Entscheidungen nicht den Interessen der Person entsprechen (z.B. Rauchen, ideelle Ablehnung von Medizin, etc.) stellt sich die \textit{paternalistische} Herausforderung; Sind die Interessen einer Person nicht gewichtiger als ihre Autonomie?

Gemäss dem Paternalismus, sollte man manchmal Menschen daran hindern, Dinge zu tun, die ihnen selber schaden. Denn (gemäss Sarah Conly) sind wir oft nicht in der Lage, unsere Ziele, aufgrund kognitiver Verzerrung oder weil wir nicht in der Lage sind, uns deren Folgen vor Augen zu führen, richtig zu verfolgen. Durch Eingriff kann die Zielerreichung anderer gefördert werden. Dies ist erlaubt, solange ein Eingriff nur zugunsten der Ziele der Person (Orientierung an den Zielen der Person) und zugunsten dem Guten für die Person (Orientierung am objektiv Guten) ist. 

\paragraph{Paternalismus} bezeichnet die Einstellung, dass man Leute daran hindern sollte, Dinge zu tun, die ihren eigenen Interessen zuwiderlaufen. Für die Autonomie bedeutet das, dass man nur autonome Entscheidungen respektiert, die im Sinne der Interessen der jeweiligen Person 
\paragraph{Paternalistisch handelt} tut man dann, wenn man a) eine Person dazu bringen will, etwas zu tun, das sie von sich aus nicht tut und b) das mit der Absicht tut, etwas Gutes zu tun.
\paragraph{Anti-paternalistisch} ist, wenn \textit{alle} autonomen Entscheidungen respektiert werden.

\subsubsection{Menschen gegen ihren Willen zu ihrem Glück bringen}
Es gibt folgende Methoden, wie man Menschen zu ihrem Glück führen kann:
\begin{enumerate}[label=(\alph*)]
	\item Durch Ausübung von Zwang
	\item Durch Manipulation (Täuschung)
	\item Durch Nudging. Das bezeichnet die Schaffung von Anreizen, die die Wahlarchitektur verändern (z.B. Werbung für gute Dinge). Dies ist, was man \textit{libertärer Paternalismus} bezeichnet. 
\end{enumerate}

\end{document}